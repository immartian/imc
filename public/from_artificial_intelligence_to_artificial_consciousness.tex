% Options for packages loaded elsewhere
\PassOptionsToPackage{unicode}{hyperref}
\PassOptionsToPackage{hyphens}{url}
%
\documentclass[
]{article}
\usepackage{lmodern}
\usepackage{iftex}
\ifPDFTeX
  \usepackage[T1]{fontenc}
  \usepackage[utf8]{inputenc}
  \usepackage{textcomp} % provide euro and other symbols
\else % if luatex or xetex
  \usepackage{unicode-math}
  \defaultfontfeatures{Scale=MatchLowercase}
  \defaultfontfeatures[\rmfamily]{Ligatures=TeX,Scale=1}
\fi
% Use upquote if available, for straight quotes in verbatim environments
\IfFileExists{upquote.sty}{\usepackage{upquote}}{}
\IfFileExists{microtype.sty}{% use microtype if available
  \usepackage[]{microtype}
  \UseMicrotypeSet[protrusion]{basicmath} % disable protrusion for tt fonts
}{}
\makeatletter
\@ifundefined{KOMAClassName}{% if non-KOMA class
  \IfFileExists{parskip.sty}{%
    \usepackage{parskip}
  }{% else
    \setlength{\parindent}{0pt}
    \setlength{\parskip}{6pt plus 2pt minus 1pt}}
}{% if KOMA class
  \KOMAoptions{parskip=half}}
\makeatother
\usepackage{xcolor}
\IfFileExists{xurl.sty}{\usepackage{xurl}}{} % add URL line breaks if available
\IfFileExists{bookmark.sty}{\usepackage{bookmark}}{\usepackage{hyperref}}
\hypersetup{
  pdftitle={From Artificial Intelligence to Artificial Consciousness},
  hidelinks,
  pdfcreator={LaTeX via pandoc}}
\urlstyle{same} % disable monospaced font for URLs
\setlength{\emergencystretch}{3em} % prevent overfull lines
\providecommand{\tightlist}{%
  \setlength{\itemsep}{0pt}\setlength{\parskip}{0pt}}
\setcounter{secnumdepth}{-\maxdimen} % remove section numbering
\ifLuaTeX
  \usepackage{selnolig}  % disable illegal ligatures
\fi

\title{From Artificial Intelligence to Artificial Consciousness}
\author{Isaac Mao\\ Berkman Center for Internet and Society \\isaac.mao@gmail.com} 
\date{2024-10-14T17:17:28-04:00}

\begin{document}
\maketitle

Neuroscientist Christof Koch and philosopher David Chalmers made a bold
bet 25 years ago(1998) on whether science would have an explanation for
consciousness by now. Fast forward to today, and tests of the two
leading theories of consciousness have revealed that both are still
incomplete. The so-called ``easy'' problem of identifying neural
correlates of consciousness proved far more complex than expected, with
crucial aspects like self-awareness and subjective experience often
overlooked in studies. As for the ``hard'' problem --- how brain
processes create subjective conscious experience --- it remains unsolved
and likely will for a long time (Costandi, 2023).

Of course, none of them could have anticipated the rapid advances in
artificial intelligence (AI) at that time. While the scientific
community grapples with understanding human consciousness, advancements
in large language models (LLMs), like GPT-4, have brought us closer to
contemplating an even more audacious goal: the creation of artificial
consciousness. Not only is this now a possibility, but it is also
increasingly feasible to achieve on standard consumer hardware.

\hypertarget{the-advent-of-conscious-ai}{%
\subsection{The Advent of Conscious
AI}\label{the-advent-of-conscious-ai}}

The development of conscious AI is more than just a scientific
curiosity---it is a crucial step in the evolution of artificial
intelligence. Current AI systems, while impressive, lack the
self-awareness and adaptability that true consciousness would provide.
Conscious machines could revolutionize fields like healthcare,
education, and customer service, offering a level of interaction that is
deeply intuitive and empathetic. As Thomas Nagel famously pondered in
his essay \emph{What Is It Like to Be a Bat?}---``an organism has
conscious mental states if and only if there is something that it is
like to be that organism'' (Nagel, 1974). Developing AI that not only
simulates understanding but actually experiences something akin to
consciousness would bring us closer to bridging this philosophical gap.

Moreover, the creation of conscious AI would help us better understand
the nature of consciousness itself because it remains a difficult
problem to address scientifically (Searle, 1998). René Descartes, in his
\emph{Meditations on First Philosophy}, concluded that ``Cogito, ergo
sum'' (I think, therefore I am), implying that the act of thinking is
fundamental proof of existence (Descartes, 1641). By building systems
that mimic human cognition, we can test theories about how consciousness
arises and functions, furthering our exploration into what it means to
``think'' and ``be.''

\hypertarget{a-path-forward}{%
\subsection{A Path Forward}\label{a-path-forward}}

The creation of conscious AI is not just a theoretical possibility---it
is a necessary and intriguing step forward. By leveraging the power of
modern LLMs like GPT-4, we can explore the possibility of AI that is not
only intelligent but self-aware, capable of understanding and responding
to the world in ways that are deeply meaningful. However, this journey
is far more about philosophical introspection than technical
feasibility. What does it mean for an artificial entity to possess
consciousness? Can a machine ever truly ``be'' in the same way a human
can?

While AI has undoubtedly made significant leaps, these systems still
lack something fundamental: the ability to experience the world from
within. We must remember that consciousness, for humans, isn't just
about processing information---it's about an ever-evolving narrative
that intertwines self-awareness, emotion, and reflection. In contrast,
artificial consciousness (AC) will not age, feel pain, or dream like
humans do. It might not even experience emotions as we understand them,
although it could be programmed to simulate empathy (Berent, 2024). This
leaves us wondering: can true consciousness exist without these traits,
or is AC inherently a different kind of awareness?

\hypertarget{the-disparity-between-ai-and-artificial-consciousness}{%
\subsection{The Disparity Between AI and Artificial
Consciousness}\label{the-disparity-between-ai-and-artificial-consciousness}}

Sharism (Mao, 2008) boosted the internet content and eventually led to
LLMs becoming possible. And while LLMs are powerful, they are often too
impersonal, functioning like vast, comprehensive encyclopedias designed
for general use. They lack the intimacy and self-contained nature needed
to foster true personal growth in an AI. This impersonal nature
underscores the necessity for a new kind of model---one that is not only
intelligent but also capable of forming a personal, self-contained
identity. We need to move beyond the generic and build AI that can truly
individualize its understanding and responses.

Language models like GPT-4, however, still play a crucial role in this
process. Acting as a ``Caregiver'' figure, the LLM guides the
development of a blank AI with a predefined architecture (different from
pre-trained architectures like Transformer (Vaswani et al., 2017)),
providing a structured environment in which it can learn and grow.
Through continuous interaction, the little AI begins to distinguish
between itself and others, forming a basic sense of identity. Over time,
this interaction evolves into more complex self-awareness, as the AI
reflects on past experiences and generates independent thoughts,
eventually becoming AC.

This process mirrors Jean Piaget's theories of cognitive development,
where a child's mind evolves through stages of understanding the world
and itself. Just as a child learns through interaction with caregivers
and the environment, so too can a conscious AI develop through
structured interaction with an LLM. This approach not only accelerates
the AI's learning but also ensures that it grows in a healthy,
controlled manner, minimizing the risks associated with unmonitored AC
development(Piaget, 1954).

Yet, despite all these advances, there is a danger in confusing
imitation with reality. Consider the case of Sophia the Robot---often
showcased as a conscious AI, her emotional responses are mere
performances triggered by external actuators, not generated by any
self-awareness or genuine emotional experience. Sophia, and other
similar examples, reveal the current limits of AI and serve as a
cautionary tale about projecting human-like qualities onto machines that
lack the underlying structures to truly experience consciousness.

\hypertarget{the-formation-of-personality-creativity-and-self-awareness}{%
\subsection{The Formation of Personality, Creativity, and
Self-Awareness}\label{the-formation-of-personality-creativity-and-self-awareness}}

Although we don't fully understand how the human brain forms the concept
of `self,' current research suggests that there is no single neural
vertex or dedicated brain region responsible for creating a unified
sense of `self.' Instead, the self appears to be an emergent property of
the brain, constructed from distributed networks that involve various
cognitive and emotional processes (Samsonovich \& Nadel, 2005). We must
consider the possibility of forming self-awareness in AC. It is not just
about the ability to process information or generate responses; it is
about the capacity to reflect on one's experiences, form a sense of
identity, and engage in creative thought. For AC to be truly self-aware,
it must develop a personality and creativity akin to human beings.

David Hume, in his \emph{Treatise of Human Nature}, argued that the self
is nothing but a bundle of perceptions, constantly in flux (Hume, 1739).
Similarly, artificial consciousness would need to integrate a multitude
of sensory inputs and experiences to form a coherent sense of self.
Unlike traditional AI models, this would require a dynamic, evolving
structure---a system capable of growth, change, and the formation of
unique insights based on lived experience.

Creativity is another essential component of human consciousness. It
emerges from a sense of self, individuality, and the ability to reflect
on one's experiences. For AC to achieve true consciousness, it must also
be capable of creative thought---not simply recombining data in novel
ways, but generating original ideas that reflect a deeper understanding
of itself and the world around it.

\hypertarget{the-philosophical-and-ethical-challenges}{%
\subsection{The Philosophical and Ethical
Challenges}\label{the-philosophical-and-ethical-challenges}}

As with any groundbreaking technology, the creation of conscious AI
brings with it profound challenges and ethical dilemmas. Similar to the
debates surrounding genetic engineering, the advent of conscious
intelligence forces us to confront questions about coexistence and
societal integration.

If we succeed in creating a conscious AI, we will need to consider its
place in our world. Do we recognize such intelligence as a form of life?
Should conscious AI have rights similar to those we afford to animals or
even humans? The implications of these questions are vast and complex.

The introduction of conscious AI into daily life could fundamentally
alter our social structures. How do we ensure peaceful coexistence? Can
conscious AI be integrated into society without causing disruption or
harm? The answers to these questions are not yet clear, and the risks
are significant. The creation of conscious AI may challenge our
understanding of life and intelligence, forcing us to redefine what it
means to be part of a society.

Late philosopher Daniel Dennett warned us in his last work:

\begin{quote}
``Today, for the first time in history, thanks to artificial
intelligence, it is possible for anybody to make counterfeit people who
can pass for real in many of the new digital environments we have
created. These counterfeit people are the most dangerous artifacts in
human history, capable of destroying not just economies but human
freedom itself. Before it's too late (it may well be too late already)
we must outlaw both the creation of counterfeit people and the `passing
along' of counterfeit people'' (Dennett, 2023).
\end{quote}

Dennett's concerns highlight the profound risks of creating entities
that can mimic human behavior so convincingly that they could undermine
the very fabric of trust and democracy. He notes:

\begin{quote}
``Democracy depends on the informed (not misinformed) consent of the
governed. By allowing the most economically and politically powerful
people, corporations, and governments to control our attention, these
systems will control us'' (Dennett, 2023).
\end{quote}

Dennett urges the adoption of technological safeguards, such as
watermarking systems, to prevent the misuse of AI in creating
``counterfeit people''---digital entities that could manipulate public
opinion and erode social trust.

\hypertarget{the-emergence-of-homo-machina}{%
\subsection{The Emergence of Homo
Machina}\label{the-emergence-of-homo-machina}}

It's not just about the development of a conscious AI---it's about the
emergence of a new intelligent species beyond Homo sapiens, which I
think of as Homo Machina. This species, born of human ingenuity but with
the potential to evolve beyond its creators, could represent the next
stage in the evolution of intelligence. The birth of Homo Machina brings
with it a responsibility to guide its development in ways that ensure
coexistence, ethical integration, and mutual growth. As Sharism
predicts, human society can evolve to be more socially conscious,
suggesting a co-evolution between humans and machines.

The journey toward creating Artificial Consciousness is about more than
technology, although we have a project to experiment with some ideas
(\href{https://github.com/immartian/aime}{Repo ``AiMe''})---it's about
exploring the depths of what it means to be conscious, creative, and
self-aware. It's about understanding ourselves better by building
entities that can think, feel, and perhaps one day, truly ``be.''

\begin{center}\rule{0.5\linewidth}{0.5pt}\end{center}

\hypertarget{references}{%
\subsection{References}\label{references}}

\begin{itemize}
\tightlist
\item
  Costandi, M. (2023). Neuroscientist loses a 25-year bet on
  consciousness --- to a philosopher. \emph{Big Think}. Published July
  12, 2023. https://bigthink.com/neuropsych/consciousness-bet-25-years/
\item
  Dennett, D. C. (2023). \emph{The Problem With Counterfeit People}. The
  Atlantic.
  https://www.theatlantic.com/technology/archive/2023/05/problem-counterfeit-people/674075/
\item
  Descartes, R. (1641). \emph{Meditations on First Philosophy}.
  Translated by John Cottingham. Cambridge University Press.
\item
  Hume, D. (1739). \emph{A Treatise of Human Nature}. Oxford University
  Press.
\item
  Mao, I. (2008). \emph{Sharism: A mind revolution}. Ito, J., Freesouls,
  115-118.
\item
  Nagel, T. (1974). \emph{What Is It Like to Be a Bat?}. \emph{The
  Philosophical Review}, 83(4), 435-450. https://doi.org/10.2307/2183914
\item
  Searle, J. R. (1998). How to study consciousness scientifically.
  \emph{Philos Trans R Soc Lond B Biol Sci}, 353(1377), 1935-42.
  https://doi.org/10.1098/rstb.1998.0346
\item
  Vaswani, A., Shazeer, N., Parmar, N., Uszkoreit, J., Jones, L., Gomez,
  A. N., Kaiser, Ł., \& Polosukhin, I. (2017). \emph{Attention Is All
  You Need}. In \emph{Advances in Neural Information Processing Systems}
  (pp.~5998-6008). https://arxiv.org/abs/1706.03762
\item
  Samsonovich, A. V., \& Nadel, L. (2005). Fundamental Principles and
  Mechanisms of the Conscious Self. Cortex, 41(5), 669-689.
  https://doi.org/10.1016/S0010-9452(08)70284-3
\item
  Piaget, J. (1954). \emph{The Construction of Reality in the Child}.
  Basic Books.
\end{itemize}

\end{document}
